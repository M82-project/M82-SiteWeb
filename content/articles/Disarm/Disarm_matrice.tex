En octobre 2022, l’entreprise de cybersécurité américaine Mandiant a publié un article décrivant plusieurs campagnes d’influences semble-t-il d’origine chinoise, visant les États-Unis. Les éléments relevés décrivaient trois narratifs différents portés par le même acteur.

Cet acteur, Dragonbridge, a été observé dès 2019 par Mandiant qui a constaté de nombreuses campagnes d’influence portées par ce réseau de milliers de comptes présents sur de nombreux réseaux sociaux et canaux de communication. Si, au départ, ce groupe a surtout mené des campagnes d’influences en faveur de la Chine, il s’attaque désormais, depuis quelques mois, à l’image des États-Unis. Il a également utilisé ses comptes de réseaux sociaux pour lancer des campagnes de dénigrement envers des entreprises d’exploitation de terres rares, canadiennes, australiennes et américaines.

Dragonbridge a porté, cette fois-ci, trois narratifs spécifiques :

    tout d’abord, il a réattribué aux États-Unis, une campagne APT, normalement attribuée à un acteur proche de l’État chinois. En effet, en 2020, pendant la pandémie, un groupe nommé APT 41 (Advanced Persistant Threat) avait mis en place une très large campagne de cyber-espionnage ;

    il a également poussé des narratifs visant à discréditer le système électoral américain en vue des midterms ;

    enfin, il allègue que l’explosion du pipeline NordStream 2 serait dû aux États-Unis.

La publication de Mandiant revient ensuite sur toutes les tactiques, techniques et procédures (TTPs) utilisées par Dragonbridge pour pousser ces narratifs, en souligne à quel point ceux-ci sont innovants (cf. le titre de l’article : « Pro-PRC DRAGONBRIDGE Influence Campaign Leverages New TTPs to Aggressively Target U.S. Interests, Including Midterm Elections »). 

Cependant, lorsque l’on étudie ces tactiques, techniques et procédures à l’aune de la matrice Disarm on se rend compte que, même s’ils peuvent être nouveaux dans le cadre de leur exploitation par Dragonbridge, tous les éléments sont déjà présents dans Disarm.

Disarm est une matrice open-source basée sur le comportement des acteurs malveillants qui permet de visualiser et de traduire une campagne d’influence sous la forme de TTPs. Cette traduction permet, à l’instar de la matrice MITRE ATT&CK utilisée dans le cadre de la Cyber Threat Intelligence (CTI, collecte et capitalisation de renseignements sur les campagnes d’attaques cyber) d’enregistrer ces éléments et d’alimenter des outils d’archivage et d’exploitation de type MISP ou bien encore OpenCTI.

La matrice Disarm présente de nombreux intérêts dans la description de campagnes d’influence ou d’opérations informationnelles. La description des TTPs permet de mesurer, dans un premier temps, le niveau d’effort que l’acteur consent dans une campagne en cours ou qu’il cherche à mettre en place. Après avoir traduit les éléments visibles de cette campagne selon les critères de la matrice, l’analyste est en mesure de déterminer le temps investi par l’attaquant pour développer les artefacts nécessaires à sa campagne, ainsi que les moyens à sa disposition (notamment financiers). 

Dans un second temps, cette traduction permet de comprendre la structuration de cette campagne en mettant en lumière le comportement de l’attaquant. En effet, les TTPs sont liées entre elles, en terme de temporalité ou de techniques. Rendre visible une technique, tactique ou procédure permet de comprendre ce dont l’acteur a eu besoin, en amont, pour la développer et ce qu’il cherche à obtenir comme effet grâce à celle-ci. À l’instar de la CTI, l’analyse de ces TTPs constitue donc un premier pas dans la création d’outils et de méthodes pour répondre aux opérations de campagnes de manipulation de l'information.

Cet article met enfin, un élément très important en lumière. En effet, Mandiant souligne que cet acteur a déjà été observé. Les supports qu’il utilise ont déjà été relevés (profils de réseaux sociaux notamment). Pourtant il persiste en exploitant les même outils, les mêmes comptes. Ce qui souligne le fait que les acteurs de ces campagnes de manipulation de l'information ne s’arrêtent pas parce qu’on les a exposés. Ils reviennent.

Ci-dessous les différentes TTPs relevées par Mandiant dans son article et traduites dans la matrice Disarm.

TTPs : TA13 T0072.005 

Nom : Target audience analysis Political segmentation

Extrait de l’article : « Aggressively targeting the United States by seeking to sow division both between the U.S. and its allies and within the U.S. political system itself »

Cibler les États-Unis de manière agressive en cherchant à semer la division tant entre les États-Unis et leurs alliés qu'au sein même du système politique américain. Le groupe Dragonbridge utilise des narratifs existants sur la justice sociale et les conflits raciaux et s'appuie sur des narratifs comme « Le vote ne soignera pas la maladie dont souffre les États-Unis ; le système législatif américain est inefficace. »

TTPs : TA14 T0068

Nom : Respond to breaking news event or active crisis

Extrait : « Allegations that the U.S. was responsible for the Nord Stream gas pipeline explosions.

DRAGONBRIDGE’s messaging mirrored Russian President Vladimir Putin’s statements that the U.S. had sabotaged the pipelines ».

Proclamer que les États-Unis sont responsables des explosions du gazoduc Nord Stream.

Le message de DRAGONBRIDGE reflète les déclarations du président russe Vladimir Poutine selon lesquelles les États-Unis auraient saboté les pipelines.

TTPs : TA14 T0083  

Nom : Integrate target audiance vulnerabilities into narrative

Extrait : « Discredit the U.S. democratic process, including attempts to discourage Americans from voting in the 2022 U.S. midterm elections. »

Discréditer le processus démocratique américain, notamment en tentant de décourager les Américains de voter lors des élections de mi-mandat de 2022. Le narratif poussé étant que le processus démocratique américain serait en train de se détériorer, deviendrait inefficace, que la société serait fondamentalement divisée.  

TTPs : TA06 T0019.002

Nom : Hijack hashtags

Extrait : « Accounts also used the hashtags #AllRoadsLeadToChengdu or #Chengdu404, which were used by the legitimate Intrusion Truth regarding APT41 ».

Réutiliser un # employé par Intrusion Truth pour attribuer APT41 à la Chine et l’exploiter en disant que ce sont, en fait, les États-Unis qui sont derrière ce groupe.

TTPs : TA06 T0023

Nom : Distort facts	

Extrait : « Claims that the China-nexus threat group APT41 is instead a U.S. government-backed actor »

Affirmer que le groupe de menace APT41, lié à la Chine, est un acteur soutenu par le gouvernement américain.	

TTPs : TA06 T0087 & TA07 T0105.002 & TA17 T0119

Nom : Develop Video-based Content & Video Sharing & Cross-posting

Extrait : « DRAGONBRIDGE accounts posted an English-language video across multiple platforms containing content attempting to discourage Americans from voting in the upcoming U.S. midterm elections »

Des comptes DRAGONBRIDGE ont publié une vidéo en anglais sur plusieurs plateformes. Le contenu de cette vidéo vise à décourager les Américains de voter lors des prochaines élections de mi-mandat aux États-Unis.

Remarque : il est logique qu’avant de publier une vidéo, il faille la produire. L’intérêt de le faire apparaître dans la matrice c’est que l’on peut ainsi se situer temporellement dans la campagne. Produire une vidéo, commencer à la diffuser, faire en sorte que celle-ci se retrouve sur de nombreuses plate-formes pour être sûr de sa visibilité : tous ces éléments sont liés à des acteurs, des temps d’action et de planification différents. Visualiser cette TTP au sein de la matrice permet également de se poser la question des différents intervenants dans une campagne. En effet, la fabrication de la vidéo a sûrement été externalisée à une entreprise tierce.

TTPs : TA06 T0089.002

Nom : Create inauthentic documents

Extrait : « While we have previously observed DRAGONBRIDGE themes involving alleged malicious U.S. cyber activity, fabrications regarding APT41 as American in origin appears to be an escalation in the degree of implied U.S. operations. »

Créer des preuves attribuant de manière fallacieuse l'APT 41 aux États-Unis.

TTPs : TA06 T0089.003

Nom : Alter authentic documents

Extrait : « Plagiarism and Alteration of News Articles »

Plagiat et altération d'articles d'actualité : plagier des articles et des tweets existants, originellement publiés par Intrusion Truth.

TTPs : TA16 T0090

Nom : Create inauthentic accounts

Extrait : 

    « Accounts' use of profile photos appropriated from various online sources, including stock photography. 

    Suggesting that they sought to obfuscate their identities. 

    Clustering of their creation dates. 

    Suggesting possible batch creation. 

    Similar patterns in usernames consisting of English-language names, followed by seemingly random numeric strings. 

    Many accounts posting similar or identical content »


    Utilisation par les comptes de photos de profil provenant de diverses sources en ligne, y compris de photographies de stock suggérant qu'ils ont cherché à dissimuler leur identité.

    Regroupement de leurs dates de création suggérant une possible création par lots.

    Schémas similaires dans les noms d'utilisateur, composés de noms en langue anglaise, suivis de chaînes numériques apparemment aléatoires.

    Comptes publiant des contenus similaires ou identiques.


TTPs : TA16 T0099 & TA16 T0100

Nom : Prepare assets impersonting legitimate entities & Co-opte trusted sources

Extrait : « Nuanced Impersonation of Cyber Actors. We identified what we assessed with moderate to high confidence, on a per-account basis, to be eight Twitter accounts impersonating Intrusion Truth comprising part of the DRAGONBRIDGE campaign. »

    Usurpation d'identité  des cyberacteurs. Huit comptes Twitter se feraient passer pour Intrusion Truth dans le cadre de la campagne DRAGONBRIDGE. 

    Usurpation d'identité d'un groupe qui publie normalement des documents et des analyses sur la menace cyber. 

    Création de faux profils reprenant des acteurs de ce groupe pour poster des tweets plagiés ou altérés.

    Utilisation des comptes de réseaux sociaux ressemblant à un groupe connu et référencé (Intrusion Truth), utilisation de médias reconnus : un article de blog de Mandiant, un article du site d'infos Sing Tao Daily, etc.


TTPs : TA09 T0116

Nom : Comment or reply on content

Extrait : « Separate DRAGONBRIDGE accounts have also replied to tweets posted by the original Intrusion Truth, questioning the veracity of the group’s information while highlighting alleged malicious U.S. cyber activities. »

Des comptes distincts de DRAGONBRIDGE ont également répondu à des tweets postés originellement par Intrusion Truth, mettant en doute la véracité des informations du groupe tout en soulignant que les États-Unis seraient responsables d'activités cybernétiques malveillantes. 

TTPs : TA11 T0059

Nom : Play the long game

Extrait : « Several of these impersonator accounts promoted content and hashtags similar, or identical to, other DRAGONBRIDGE messaging on alleged malicious cyber activity »

Réutilisation des faux profils de réseaux sociaux pour promouvoir d'autres campagnes de Dragonbridge.

Remarque : Si certaines TTPs semblent faire doublon ou répéter les mêmes éléments, on  observe cependant que, dans la globalité de la matrice, elles sont en fait complémentaires. Finalement, c’est la place qu’elles prennent dans la temporalité de la campagne qui explique ces ressemblances.
