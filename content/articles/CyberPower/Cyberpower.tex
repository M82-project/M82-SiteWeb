\documentclass[a4paper]{article}

\usepackage[utf8]{inputenc}   
\usepackage[T1]{fontenc}      
\usepackage{geometry}  
\usepackage{hyperref}       
\usepackage[francais]{babel}  
                       
\title{Cyberpower - Eviata Matania}          
\author{Nicolas Chevrier}
\date{26 septembre 2022}                     
			    
\sloppy  
\begin{document}

À l’invitation de l’IHEDN, Eviatar Matania a pu présenter son dernier livre, édité en français, Cyberpower (Israël, la révolution cyber et le monde de demain) aux éditions Les Arènes.

Eviatar est un précurseur des questions de la cyber sécurité et a successivement occupé les fonctions de responsable du bureau national cyber (2011) puis fondateur et directeur de l’Agence nationale de sécurité de l’État d’Israël (2015).

Si l’on a tendance a souvent associer les termes d’affrontement ou de guerre au domaine cyber, Eviatar a choisi d’explorer ce qui fait d’une nation une puissance cyber. Inspiré de sa propre expérience et de la dynamique qu’il a imprimé à l’état d’Israël, il évoque des principes tels que le rôle primordial de la cyberdéfense (la meilleure défense… c’est bien la défense) et le rôle prépondérant des agences de renseignement techniques pour la maîtrise et l’emploi de l’arme cyber.

Dans un monde où la perception de l’arme cyber a basculé en 2010, pour devenir une arme d’emploi avec le ver Stuxnet, Eviatar s’attache à décrire les grands principes autour desquels les nations pourraient constituer et exercer leur puissance dans ce domaine. S’il évoque l’absence de frontières géographiques, il insiste surtout la nécessité de se défendre contre tous (et pas seulement ses voisins) ainsi que le rôle que le secteur privé comme les états doivent jouer dans ce domaine. On y retrouve des principes simples tels que le durcissement des systèmes, la développement de la résilience et de la défense nationale mais qui prennent plus de sens au travers de l’expérience de l’auteur.

Découvrez l'intégralité de \href{la conférence ici}{https://youtu.be/MEaIojimLJ}
\end{document}

