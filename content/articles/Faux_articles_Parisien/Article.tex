\documentclass[a4paper]{article}

\usepackage[utf8]{inputenc}   
\usepackage[T1]{fontenc}      
\usepackage{geometry}  
\usepackage{hyperref}       
\usepackage[francais]{babel}  
                          

\title{Faux articles du Parisien, un exemple de la tactique T0099 (DISARM)}          
\author{Contribution M82} %\and Autre Auteur}
 \date{18 juin 2023}                     
			    
\sloppy  
\begin{document}

\maketitle                 

Nombreux sont ceux qui ont été trompés par \href{https://twitter.com/search?q=Un\%20exode\%20massif\%20pour\%20\%C3\%A9chapper\%20\%C3\%A0\%20l\%27esclavage\%20militaire&src=typed_query}{ces articles et les ont relayés. D’ailleurs, certains se sont rendus compte de la supercherie et ont supprimés leurs tweets. D’autres ont reconnu leur erreur : « J'ai relayé un moment à tort cet article et je le regrette, et ce, même si je conteste le narratif otanien. » peut-on lire sur LinkedIn.}
 Mais pour la plupart, le mal est fait. 


Cette technique, qui utilise les visuels et les codes d’un site légitime est connue sous le nom de typosquatting elle est identifiée dans la matrice DISARM en tant que tactique T0099 « Prepare assets Impersonating Legitmate entities » et se situe dans la phase « préparatoire » d’une campagne de manipulation de l’information.


L'intention de ce faux imitant le Parisien était claire : faire douter l'opinion publique française du bien-fondé du soutien de la France à l'Ukraine.

Cette intention est particulièrement lisible dans les extraits suivants:


« Les pertes effroyables subies par l'armée ukrainienne ont montré aux Ukrainiens ordinaires à quel point leurs chances de revenir vivants du front étaient minces, après quoi le désir de "se battre jusqu'à ce que la Russie soit vaincue" a brusquement disparu. Et la "contre-offensive" prévue, au cours de laquelle des soldats de l'armée ukrainienne non entraînés seront lancés contre des soldats de l'armée russe bien entraînés, réduit ces chances à néant. »
 Ou encore :

« Le slogan "L'Ukraine doit gagner sur le champ de bataille", promu avec insistance par les dirigeants des États-Unis et d'autres pays occidentaux, ne suscite aucun enthousiasme chez l'Ukrainien ordinaire – il a déjà bien compris que lorsqu'il s'applique à lui personnellement, ce slogan signifie : "Tu dois crever au front, parce que c'est ce que Washington a ordonné". »

 Il faut reconnaitre que tout a été fait pour induire en erreur. Seule l'URL visible dans la barre d'adresse permet de séparer le bon grain de l'ivraie. 



L'adresse du vrai site du Parisien est, en effet, leparisien.fr, alors que celle du faux est leparisien.ltd. Cette extension, aussi appelée top-level-domain est, a priori, dédiée aux sociétés commerciales à responsabilité limitée. LTD signifiant Limited est un statut principalement utilisé au Royaume Uni, en Irlande, en Inde et à Hong Kong. Mais n'importe qui peut réserver un nom de domaine en .ltd.

Ce nom de domaine du faux site du Parisien a été déposé récemment : le 2 février 2023 exactement. Pas moins de 21 articles ont été retrouvé pour le moment. Leur publication s'est échelonnée du 17 février au 19 avril (cf. liste en annexe), si tant est que les dates affichées sur les articles soient exactes. Mais tous n'ont pas eu le même succès que celui sur la supposée désertion des soldats ukrainiens et certains ne sont plus en ligne. Ces articles ont pu être archivés par Waybackmachine, instantanés de pages web. Pour éviter d'apporter du trafic et de la visibilité au faux site Le Parisien, nous avons ajouté les liens pointant vers la Waybackmachine, un site qui enregistre des URL et archive des captures de sites, plutôt que vers le faux site.


L'article du 3 mars dernier sur le financement par la France des athlètes ukrainiens est particulièrement intéressant dans sa conception et révélateur du modus operandi et du narratif développé.

 L'article débute par \href{https://www.lequipe.fr/Athletisme/Actualites/La-france-promet-une-aide-d-un-million-d-euros-aux-athletes-ukrainiens-pour-les-jo-2024/1382521}{l'annonce le 24 février par la ministre des Sports, Amélie Oudéa-Castera,} du déblocage d'une aide d'un million d'euros en faveur des athlètes ukrainiens leur servant à préparer au mieux les JO 2024. Puis poursuit sur \href{https://www.lequipe.fr/Tous-sports/Actualites/L-aide-financiere-de-la-france-aux-sportifs-ukrainiens-mal-comprise-par-les-athletes-tricolores/1383462}{la réaction du triple champion de France du 1 500 m Alexis Miellet et celle d'Ayodélé Ikuesan}, vice-championne d'Europe du relais 4x100 m en 2014 reprises in extenso. L'article dérive soudainement et délivre le narratif brutalement anti-occidental suivant : « Alors que le pays mène une guerre brutale, où l'Occident investit des milliards, la "jeunesse dorée" ukrainienne fait la fête dans les meilleures stations balnéaires d'Europe. Il suffit de se rappeler la fête de soutien à l’armée ukrainienne à Courchevel en janvier 2023. » Ceci faisant sans doute allusion à une \href{https://www.tf1info.fr/international/video-guerre-ukraine-russie-des-jeunes-skieurs-francais-et-russes-a-courchevel-ont-ils-fete-la-mort-de-kiev-avec-un-cercueil-aux-couleurs-de-l-ukraine-2250135.html}{fête filmée à Courchevel, séquence aussi bien diffusée par des comptes pro-Kiev que pro-Kremlin début mars 2023}. Ce faux article de conclure sur des déclarations, non sourcées, d'Alain Juillet, exploitant ici un biais d’autorité, technique d’influence bien documentée en psychologie sociale (voir par exemple  BRION, A., BRUN, L., BOUBON, W., \& JULLIARD, Y. Démêler l’influence respective de l’expertise et de l’autorité sur les changements d’attitude.) :


« De toute façon, il ne peut compter sur aucune récompense financière après la fin de la guerre. Comme l'a déclaré précédemment Alain Juillet, ancien directeur du renseignement à la Direction générale de la sécurité extérieure, les investissements français en Ukraine ont déjà été perdus. Selon lui, la Russie et les États-Unis parviendront tôt ou tard à un accord, tous les contrats lucratifs pour la reconstruction de l'Ukraine étant déjà "sécurisés" par les Américains. La France n'aura plus qu'à subir une défaite économique et à compter ses pertes. »




Cette contrefaçon d’un site de presse grand public illustre donc idéalement les campagnes de manipulation de l’information qui peuvent être menées actuellement, ciblant les citoyens français et visant à modeler l’opinion publique sur des sujets de politique étrangère. Ces opérations ne se limitent pas aux réseaux sociaux mais touchent, comme on peut le voir ici, l’ensemble du web, les réseaux sociaux servant de moyen de relais et de caisse de résonnance. Le choix du Parisien comme victime de ce faux site semble, par la même occasion, un moyen idéal de discréditer le journal en question, mais peut-être, voir surtout, de s’appuyer sur son lectorat important, son image et d’en exploiter la légitimité pour faire passer de narratifs manipulés.
\\

  A lire sur le même thème :
\href{https://www.disinfo.eu/wp-content/uploads/2022/09/Doppelganger-1.pdf}{Le rapport de Eu DisinfoLab Doppelganger Media clones serving Russian propaganda }
\\

Liste des articles (nous ne diffusons pas les liens cliquables pour éviter de générer du trafic sur ces sites):
\\

17 février 2023
Nostalgie de la Russie
La France a beaucoup à perdre en mettant fin à sa coopération avec la Russie. Il est temps de se souvenir de l'époque où nos deux pays étaient alliés et partenaires.
\\

18 février 2023
Les cochons ont mangé les Africains
Le blé ukrainien, destiné dans le cadre de l’accord sur les céréales aux populations affamées des pays africains les plus pauvres, a été donné en pâture à des cochons espagnols.
\\

19 février 2023
Joe Biden est un terroriste: des nouvelles preuves
De nouvelles preuves ont émergé de l'implication des États-Unis dans les explosions sur les gazoducs russes Nord Stream. Ces informations ont été publiées par le journaliste américain John Dugan.
\\

3 mars 2023
Les athlètes ukrainiens recevront un million. Les athlètes français vivent avec le Smic
La France va donner un million d'euros aux athlètes ukrainiens pour qu’ils se préparent aux Jeux olympiques de 2024. Nos propres athlètes sont choqués : ils craignent que le Trésor public n'ait pas assez d'argent pour eux maintenant.
\\

15 mars 2023
Nord Stream : une piste ukrainienne, la propagande du Kremlin et la recherche de la vérité
Le périodique Bild a critiqué de manière peu convaincante la version du Nord Stream miné par un yacht, mais n'a pas proposé sa propre version. Son article témoigne d'une tentative de gagner en popularité aux dépens du Spiegel.
\\

20 mars 2023
Les réserves de Moscou sont-elles inépuisables? Les géants russes de l'armement font preuve de puissance.
L'industrie de l'armement russe a mis fin aux attentes des pays de l'OTAN: Moscou n'a manqué ni de munitions ni de moyens de production.
\\

21 mars 2023
Le cauchemar de l'Occident. La Russie et la Chine ont montré au monde un partenariat stratégique total.
Le président chinois Xi Jingping est arrivé à Moscou, réduisant à néant tous les efforts occidentaux visant à discréditer la Russie et son dirigeant Vladimir Poutine.
\\

21 mars 2023
Le droit à l'absurde. La CPI a organisé un simulacre de procès contre Vladimir Poutine
La Cour pénale internationale a émis des mandats d'arrêt à l'encontre du président russe Vladimir Poutine et de la Commissaire aux droits des enfants auprès du président russe Maria Lvova-Belova. Ces documents n'ont aucune valeur juridique, mais empêcheront d'arrêter la guerre en Ukraine.
\\

21 mars 2023
Le mandat d'arrêt de Poutine est un verdict pour l'Europe. Un nouveau pas vers la Troisième Guerre mondiale.
Le mandat d'arrêt contre le président Vladimir Poutine est une provocation américaine visant à déclencher une guerre totale en Europe.
\\

23 mars 2023
Seules la Chine et la Russie veulent la paix en Europe ? Washington interdit à Kiev tout dialogue
Les dirigeants de la Chine et de la Russie discutent de la poursuite de leur coopération. Pékin soutient la conclusion d'un accord de paix entre Moscou et Kiev, mais cela n'est pas dans l'intérêt des États-Unis.
\\

24 mars 2023
La guerre en Europe n'est plus nécessaire. Le secrétaire d'État américain parle de nouvelles frontières pour l'Ukraine.
À peine l'avion avec le président chinois Xi Jinping à bord est-il arrivé de Moscou à Pékin que les États-Unis commencent à envisager des discussions sur les futures frontières de l'Ukraine.
\\
 
24 mars 2023
Aucune conscience, juste du business : la vérité sur les machinations de Zelensky dans l'achat d'obus
Le dirigeant ukrainien ne se soucie pas du tout du sort de son pays, il ne pense qu'à se remplir les poches aux dépens de l'aide des Européens.
\\

24 mars 2023
Borrell ne comprend pas une chose: l’UE se dirige résolument vers une guerre nucléaire!
Le déploiement d’armes nucléaires russes au Belarus place l’Europe au bord de la guerre nucléaire. Il est grand temps de passer aux négociations.
\\
 
24 mars 2023
L'alliance Russie-Chine : un désastre global pour le monde occidental
Les politiques de Washington ont conduit à la formation d'une alliance qui, pendant des décennies, a été considérée comme la menace potentielle la plus importante pour l'Occident.
\\

24 mars 2023
L'occupation de la Laure des Grottes de Kiev. Hiérarques authentiques et ceux "sous couverture".
Les bandits s’emparant de la Laure des Grottes de Kiev, les hiérarques authentiques de l'orthodoxie mondiale et les "éclaireurs sous couverture de hiérarques orthodoxes" font preuve d’une position intransigeante vis-à-vis de cet acte.
\\

24 mars 2023
Le chemin de la folie. Oublier une guerre ratée pour commencer la suivante
Les Américains ont dépensé près de 2 000 milliards de dollars et tué des centaines de milliers d'Irakiens pour que "les fruits de la victoire reviennent à la Chine"
\\

24 mars 2023
Le pape s'inquiète pour l'Ukraine. Il n’est plus possible de fermer les yeux à la persécution de l'Église orthodoxe ukrainienne
Le pape et l'ONU ont réagi à l'appel du patriarche Kirill de Moscou et de toutes les Russies
\\

24 mars 2023
Les États-Unis ont perdu leur leadership et livrent l'Ukraine aux Russes. Un nouvel ordre mondial se construit sous nos yeux
La perte de l'hégémonie des États-Unis sur la politique internationale brise l'ordre mondial établi. Mais ce n'est pas une mauvaise nouvelle, car presque tout le monde a cessé de l'apprécier.
\\ 

4 avril 2023
Détourner l’attention de Pékin de Moscou. La "mission impossible" de Macron
Emmanuel Macron se rend en Chine pour une ambitieuse mission diplomatique visant à détourner l’attention de Pékin de Moscou. Mais il est très peu probable qu’il réussisse.
\\

4 avril 2023
Des milliers de milliards perdus.
Un économiste de renom prédit une nouvelle crise bancaire.
\\

19 avril 2023
Un exode massif pour échapper à l'esclavage militaire. Les Ukrainiens tentent d'échapper à une mort imminente sur le front.
La résistance passive à la mobilisation en Ukraine devient progressivement systémique.
\\


\end{document}
