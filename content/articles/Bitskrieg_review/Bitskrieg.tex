\documentclass[a4paper]{article}

\usepackage[utf8]{inputenc}   
\usepackage[T1]{fontenc}      
\usepackage{geometry}  
\usepackage{hyperref}       
\usepackage[francais]{babel}  
                       
\title{Bitskrieg:The New Challenge of Cyberwarfare}          
\author{Bertrand Boyer} %\and Autre Auteur}
 \date{21 septembre 2022}                     
			    
\sloppy  
\begin{document}

\maketitle                 

Cette recension a été publiée dans le numéro de printemps 2022 de Politique étrangère (n° 1/2022). Elle propose une analyse de l’ouvrage de John Arquilla, Bitskrieg: The New Challenge of Cyberwarfare (Polity Press, 2021, 240 pages). 
\\

Plus de trente ans après son article \textit{cyberwar is coming} écrit avec David Ronfeldt, John Arquilla prolonge son étude de la conflictualité à l’ère numérique en appelant à un véritable changement d’approche. Soulignant les limites de la conception d’une défense statique de type ligne Maginot, il poursuit la métaphore historique en posant le concept de \textit{bitskrieg}. John Arquilla est un auteur reconnu sur les questions de cyberdéfense et sa proximité avec les sphères du pouvoir aux États-Unis depuis plus de trente ans en fait un témoin précieux pour appréhender les approches stratégiques développées outre-Atlantique.

Il propose ici un travail qui oscille entre ses premiers constats des années 1990 et la situation actuelle. L’auteur porte un regard parfois critique sur ses propres approches et les conseils qu’il a pu donner au plus haut niveau de l’appareil sécuritaire américain, depuis la première guerre du Golfe en 1991. Prenant en compte les dangers et opportunités de la révolution de l’information, Bitskrieg cherche à dépasser la notion de \textit{cyberguerre}. Véritable fil rouge de l’ouvrage, cet aspect est illustré par des retours d’expérience et exemples qui ont ponctué les dernières décennies.
\\

Constatant la lente émergence d’une \textit{cyberguerre} destructrice et mortifère, l’auteur présente au long des cinq chapitres les récentes évolutions en matière de conflictualité et souligne en particulier l’arrivée de ce qu’il appelle la \textit{Cool War}. S’il a fallu 138 années entre l’apparition du premier sous-marin et son intégration complète dans la palette stratégique, la \textit{Cool War} s’est imposée en moins de dix ans. Cette forme d’affrontement fait aussi bien la part belle aux attaques informatiques conduites par des cybercriminels qui font peser un risque croissant sur les économies connectées, qu’au vol de données à caractère personnel, comme celui ayant touché le personnel de l’administration fédérale en 2015. Cette guerre \textit{cool}, dont les principaux acteurs ne sont pas nécessairement en uniforme, illustre la lenteur avec laquelle la cyberguerre se matérialise. En effet, alors même que la crainte d’un \textit{cyber Pearl Harbour} est communément évoquée depuis de nombreuses années, l’auteur s’interroge sur la faible incarnation dans les opérations militaires des actions cyber offensives. Son expérience personnelle lui permet de mettre en lumière les difficultés d’intégration de ces actions par l’appareil militaire, par nature conservateur.
\\

Même si la bataille numérique semble présenter un nouveau visage, l’auteur s’attache à démontrer que la cyberguerre n’est en fait qu’un sous-domaine de la guerre de l’information et de la guerre en réseau. Cet aspect est selon lui toujours mal appréhendé par les appareils politiques et militaires. Enfin, l’auteur consacre un chapitre à la question du contrôle des armes dans le cyberespace et à la difficulté de mettre en œuvre un mécanisme reposant sur autre chose qu’une politique déclaratoire.
\\

John Arquilla appelle ainsi à repenser la cybersécurité à l’aune des évolutions techniques et stratégiques. Il présente les nombreux défis qu’implique une nouvelle approche de la cyberdéfense, plus dynamique et reposant davantage sur la traque de l’adversaire. Très accessible, Bitskrieg synthétise sans concession deux décennies d’approches conceptuelles de la cyberguerre et alimente la réflexion sur l’urgence de la prise en compte du fait numérique pour prévenir le prochain choc.

\end{document}
