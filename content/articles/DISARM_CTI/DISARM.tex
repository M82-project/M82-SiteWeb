


Disarm, un pas vers la CTI pour lutter contre la désinformation
La multiplication des campagnes de désinformation et leur impact potentiel sur la société ont conduit de nombreuses organisations (universités, think tank, ONG, administrations, plateformes) à étudier et analyser cette menace. Cet intérêt a résulté dans l’élaboration de schémas descriptifs permettant de mettre en lumière le comportement de ces acteurs et les objectifs de leurs campagnes. Cette démarche a ainsi pu se nourrir de la riche littérature et de l’expérience accumulée dans un autre champ d’analyse de la menace : la Cyber Threat Intelligence (CTI).

Les éléments nécessaires à la création de campagnes de manipulation de l’information sont multiples : memes, narratifs, faux sites, et présentent des niveaux de complexités divers. A l’image des APT (Advanced Persistent Threat), les groupes qui conduisent ces opérations ont donc des besoins très spécifiques et doivent mettre en place des infrastructures et des outils en amont de leurs actions. C’est précisément cette approche qui a conduit en 2011, le groupe Lockheed Martin à mettre en place sa célèbre Cyber kill chain. Cette représentation décrit les étapes d’une attaque informatique. Ainsi, chaque étape permet de déduire les indices à détecter afin de contrer le plus en amont possible une intrusion dans un système d’information. Dans les campagnes informationnelles et typiquement celles conduites via les réseaux sociaux, cette approche est totalement transposable. 

Approche  de la désinformation dans le cadre de la CTI

Une brève histoire des modèles en CTI

La Cyber Threat Intelligence est une discipline qui vise à identifier et analyser les menaces. Elle s'intéresse évidemment à l’analyse des données techniques liées à une attaque ou à des menaces connues mais également au contexte dans son ensemble pour, comme dans le domaine militaire, « éclairer la décision ». Le renseignement sur les menaces cyber est une activité de renseignement dont le produit, l’analyse, doit orienter les décideurs sur les actions à conduire pour minimiser la menace, la prévenir ou la traiter.

La modélisation en CTI se fonde initialement sur l’analyse d’indicateurs statiques, c’est le point de départ du concept de kill chain ou encore du Diamond Model. Ce dernier élaboré à partir de 2013 par Sergio Caltagirone, Andrew Pendergast et Christophe Betz est une matrice qui s’inspire du modèle de Michael Porter utilisé en analyse macroéconomique. Dans le cadre de la CTI, la modélisation des attaques informatiques repose sur des événements séquencés pour lesquels on identifie un adversaire, des capacités déployées, une infrastructure et une victime. Ces quatre facteurs constituent les quatre angles du diamant. Ces évènements sont rassemblés pour constituer une campagne 


(fig 1 : source Threat Intel 101 — Le modèle en Diamant, Sekoia

https://medium.com/cyberthreatintel/threat-intel-101-le-mod%C3%A8le-en-diamant-81ff503ada7f). 

En analysant et en rassemblant plusieurs campagnes qui partagent certaines caractéristiques, on peut faire apparaître un « groupe d’activité ». Ces groupes sont à l’origine de la numérotation des APT (Advanced Persistant Threat) et permettent d’analyser les acteurs sur la base de leur comportement. 

Ainsi le rapport Mandiant de 2013 portant sur APT 1 marque un tournant majeur dans la jeune histoire de la Cyber et de l’analyse de la menace car il lance véritablement l’analyse « à fin d’attribution ». Les modèles développés fournissent la base nécessaire à l’attribution ou au moins à l’imputation, pour ce qui relève des « campagnes étatiques », car l’intrusion est documentée de manière globale et favorise la prise en compte du contexte et des données non-techniques. Ces données permettent de dresser une première approche de la stratégie de l’attaquant. Sur cette base, les éditeurs pointent plus ou moins directement la responsabilité vers des entités étatiques. 

Progressivement, l’analyse d’une intrusion sera documentée de façon globale et dynamique. Le modèle Diamant permet donc d’accompagner une démarche d’investigation au cours de laquelle l’analyste pivote d’une information à une autre, découvrant de nouveaux éléments sur l’attaquant. En s’appuyant sur ce formalisme, les investigateurs éclairent progressivement le contour des attaquants et de leurs comportements. Comme tous les modèles, le Diamant a ses limites et l’une des principales demeure la difficulté à le décliner en une stratégie de détection d’incidents, de collecte et d’enrichissement de logs au sein d’un système d’information. S’il est visuellement attractif sur un powerpoint destiné aux décideurs pour présenter une menace, il est rarement l’outil préféré des équipes de réponse à incidents. Pourtant compatible avec une Cyber Kill Chain les résultats sont parfois peu lisibles. Beaucoup y préfèreront le framework MITRE ATT&CK.

MITRE ATT&CK, la matrice des matrices

La démarche proposée par MITRE ATT&CK s’appuie sur une cartographie détaillée des actions que doit entreprendre un attaquant pour conduire une intrusion sur un système d’information. C’est donc une démarche centrée sur le comportement de l’attaquant et sa compréhension plus que sur les événements comme dans les modèles précédents.

MITRE est historiquement connu dans la communauté de la sécurité informatique pour maintenir la liste des Common Vulnerabilities Exposure (CVE). Depuis 2013, il développe un modèle d’analyse de la menace qui n’a cessé d’évoluer. Rendu public en 2015, cet outil très flexible se présente sous la forme d’un wiki qui rassemble des données sur les acteurs, les campagnes et les tactiques, techniques et procédures (TTPs). Véritable base de connaissance qui cartographie les modes opératoires des groupes d’attaquants (MOA), il s’impose comme une référence au sein de la communauté CTI.

Le modèle ATT&CK, qui signifie Adversarial Tactics, Techniques, and Common Knowledge, est connu pour sa visualisation sous forme de matrice qui rend l’approche extrêmement intuitive. La matrice présente en entrée les tactiques et décrit pour chacune les techniques que l‘attaquant devra mettre en œuvre. 

Les tactiques, contrairement à la définition militaire, désignent un ensemble d'objectifs que l’attaquant cherche à atteindre ou, plus précisément, doit atteindre dans le système informatique visé suivant son schéma d’attaque. Initialement au nombre de 9 puis de 11 MITRE distingue aujourd’hui les tactiques suivantes :

    Reconnaissance (10 techniques – active scanning, search open technical database, gather victim network information, etc.) ;

    Ressource development (7 techniques – acquire infrastructure, compromise accounts, etc.) ;

    Initial access (9 techniques – phising, supply chain compromise, valid accounts, etc. ;

    Execution (13 techniques – command and scriptying interpreter, native API, User execution, etc.) ;

    Persistence (19 techniques – external remove service, create account, implant internal image, etc.) ;

    Privilege escalation (13 techniques – boot or logon autostart execution, hijack execution flow, etc.) ;

    Defense evasion (42 techniques – access token manipulation, deploy container, rootkit, XSL script processing, etc.) ;

    Credential access (17 techniques – brute force, forge web credentials, input capture, etc.) ;

    Discovery (30 techniques – cloud service discovery, account discovery, domain trust discovery, etc.) ;

    Lateral movement (9 techniques – exploitaiton of remote service, lateral tool transfer, etc.) ;

    Collection (17 techniques – automated collection, audio capture, screen capture, email collection, etc.) ;

    Command and control (16 techniques – data obfuscation, proxy, remote access software, encrypted channel, etc.) ;

    Exfiltration (9 techniques – exfiltration over C2 channel, exfiltration over web service, exfiltration over physical medium, etc.) ;

    Impact (13 techniques – account removal, data destruction, data manipulation, defacement, etc.).


Chaque tactique donne lieu à une cartographie de techniques que nous retrouvons dans la matrice. Souple d’emploi, le navigateur permet de visualiser les techniques mises en œuvre par un attaquant, de versionner et d’analyser son évolution.

L’analyse des campagnes de désinformation peut ainsi très largement bénéficier de l’expérience acquise dans le domaine de la CTI. En particulier en important la notion de TTPs et en s’appuyant sur les qualités éprouvées de MITRE ATT&CK que sont : 

    la souplesse d’utilisation : la matrice est régulièrement mise à jour afin de correspondre à l’évolution des pratiques des acteurs malveillants ;

    une approche heuristique et l’association d’une base de connaissance. Les TTPs sont liés entre eux. En découvrir un c’est comprendre où il se place dans l’intégralité d’une campagne, et ainsi anticiper ce que l’on peut encore attendre ou ce qu’il peut déjà s’être passé en amont.

    la compréhension du comportement d’un acteur, en liant des détections pour mieux évaluer ou estimer les objectifs stratégiques de la campagne ;

    la qualification d’une campagne dans son ampleur, dans son coût pour l’attaquant en évaluant la difficulté pour lui de mettre en œuvre certaines techniques.

En capitalisant sur l’approche CTI et la modélisation ATT&CK, l’analyse des campagnes de désinformation peut rapidement gagner en maturité et développer des méthodes de détection et de remédiation qui reposent sur une analyse objective des techniques mises en œuvre par les attaquants. 

Étudier la désinformation

Les campagnes dmanipulation de l’information en ligne peuvent prendre de multiples formes. À titre d’exemple, une campagne d’astroturfing demande de coordonner un réseau de trolls ou de bots pour pousser un message ou un hashtag. Cela ne présente pas le même niveau de complexité, ni le même besoin en temps ou en compétences techniques que l’usage du typosquatting ou la création d’un réseau de sites internet. Par ailleurs, bien que la question des campagnes d’ingérence étrangère soit ancienne, les récents développements du web et des outils à disposition ont permis aux acteurs malveillants de disposer de ressources extrêmement variées et performantes.

Depuis 2019, de nombreux acteurs de la lutte contre la désinformation ont cherché un moyen de décrire les campagnes de manière objective et uniforme. Nombre d’entre eux se sont appuyé sur des réflexions empiriques liées à la menace informationnelle, à l’instar du Coordinated inauthentic behavior (CIB) développé par Facebook, la méthode ABC développée par Camille FRANÇOIS pour Graphika et augmentée d’un D par Alexandre ALAPHILIPPE et d’un E (Effect) par James PAMMENT. Ces modèles ont, entre autres, répondu aux problématiques spécifiques de la diffusion des campagnes sur les plateformes de réseaux sociaux.

D’autres modèles de description s’appuient sur des canevas issus de l’étude de la menace dans le champ cyber, du fait notamment de la présence d’éléments informationnels au sein d’attaques cyber. C’est le cas de la matrice Disarm qui capitalise sur le savoir-faire de la matrice MITRE ATT&CK dont elle est largement inspirée.

De quoi parle-t-on ?

Créée en 2019, la matrice AMITT est un cadre d’analyse qui permet de décrire et de comprendre les incidents (terminologie issue du cyber) de désinformation. Le Minsifosec est un groupe de travail qui a réfléchi et établi des standards pour permettre le partage de l’information dans le cadre des campagnes de désinformation. Pour ce faire, ils ont examiné différents modèles comportementaux issus de la sécurité de l’information, de l’analyse des réseaux sociaux, du marketing. Ils ont ainsi créé la matrice AMITT en prenant pour modèle le Mitre ATT&CK. Aujourd'hui mise à jour et maintenue par la Disarm Foundation, sous le nom de Disarm, la matrice s’est enrichie de tactiques et de techniques et procédures supplémentaires.

Disarm est donc une matrice de description des opérations de désinformation centrées sur le comportement de l’attaquant et décrite par l'intermédiaire de tactiques, techniques et procédures (TTP).

À l’instar du MITRE ATT&CK, Disarm présente de nombreux avantages dans le cadre de la description des campagnes. Elle permet :

    d’imputer ou, au moins, de caractériser un acteur par la récurrence des techniques, tactiques ou procédures qu’il emploie lors de ses campagnes ;

    de comprendre le niveau d’effort et des moyens à la disposition d’un acteur malveillant ;

    de capitaliser des campagnes et incidents dans un modèle stable, pérenne, interopérable et ouvert.

    de partager de l’information structurée entre les acteurs de la lutte contre les campagnes de manipulation de l’information.

Disarm

La matrice Disarm est structurée en 3 éléments principaux : phases, étapes (ou tactiques) et techniques, du plus macro au plus micro. 

Les phases

Les campagnes d’influences informationnelles sont composées en général de 4 phases qui correspondent aux séquences de mise en œuvre de la campagne. Chaque phase est le regroupement de tactiques et de leurs techniques associées. 

Figure 2. Structuration des tactiques, techniques et procédures au sein de la matrice Disarm

    La planification permet de visualiser le but de la campagne ou de l’incident. Elle définit les moyens nécessaires à sa mise en place. Cette étape se concentre sur les résultats attendus par les acteurs malveillants. Dans le domaine militaire on parle d’état final recherché (EFR).

    La préparation regroupe les activités menées avant l'exécution de la campagne : le développement d’un écosystème nécessaire pour soutenir une action (personnes, réseau, canaux, contenu, etc.).

    L’exécution consiste en la réalisation de l’action, de l'exposition initiale à la conclusion ou au maintien de la présence en cas de menace persistante (on notera ici la similitude avec les attaques informatiques ou l’on évoque la persistance comme un facteur central d’une opération). 

    L’évaluation est une étape nécessaire qui détermine l’efficacité de l’action.

Les étapes (ou tactiques)

Les phases sont découpées en tactiques. Elles sont aujourd’hui au nombre de seize et leur nombre peut changer en fonction de la mise à jour de la matrice (à l’instar d’ATT&CK) pour correspondre aux évolutions des pratiques des acteurs malveillants.

Phase
	

Tactique
	

Description de la tactique
	

Objectif de la tactique

Planification
	

Planification de la stratégie
	

Définir l’état final recherché, c’est-à-dire l’ensemble des conditions requises permettant de déclarer l’accomplissement des objectifs stratégiques.
	

Mettre en cohérence les audiences ciblées et les finalités stratégiques de la campagne.

​
	

Planification des objectifs
	

Planifier des objectifs stratégiques liés à l’exécution de tactiques nécessaires à leur réalisation.
	

Définir des objectifs intermédiaires permettant d’atteindre l’état final recherché.

​
	

Analyse des publics cibles
	

Identifier et analyser des audiences ciblées, c’est-à-dire l’ensemble de leurs attributs qu’une opération d’influence pourrait incorporer dans sa stratégie vers celles-ci.
	

Permettre la personnalisation des contenus et de la stratégie d’influence selon l’analyse obtenue.

Préparation
	

Développement des récits
	

Promouvoir et renforcer des récits généraux en s’appuyant sur de nombreux récits locaux, diffusés régulièrement (à bas bruit en général) via les différents artefacts créés pour la campagne.
	

Occuper et dominer le débat numérique en imposant progressivement des récits phares sur la société.

​
	

Fabrication des contenus
	

Créer ou acquérir des textes, images et tous les contenus nécessaires au soutien des récits généraux et des récits secondaires.
	

Soutenir la mise en place des récits phares à l’aide de contenus crédibles.

​
	

Mise en place des canaux de communication
	

Créer, modifier ou compromettre des outils de messagerie (comptes de réseaux sociaux, chaînes de médias, personnel opérationnel).
	

Faire la promotion des messages directement à l’audience ciblée sans dépendre d’entités externes.

​
	

Mise en place des canaux de légitimation des récits
	

Établir des ressources dédiées à la légitimation des récits (faux sites news, faux experts, sources vérifiées compromises).
	

La création de relais informationnels soutient la légitimation des récits.

​
	

Microciblage des audiences clefs
	

Cibler des groupes de population très spécifiques via des contenus localisés, les fonctionnalités publicitaires des plateformes ou la création de chambres d’écho.
	

Il vise à garantir une meilleure perception des récits de la part de certaines audiences et à consolider voire polariser les opinions d’audiences clefs.

​
	

Sélection des canaux selon leur usage
	

Sélectionner, après étude de marché, des vecteurs des différents narratifs ou artefacts créés, que ce soit des plateformes (réseaux sociaux, plateformes en ligne de partage de vidéos, hashtags, etc.), des médias traditionnels (télévision, journaux), etc. Étudier les fonctionnalités et l’accessibilité des plateformes.
	

Déterminer quels seront les canaux et leurs usages qui maximiseront la diffusion des narratifs et artefacts de l’opération d’influence.

Exécution
	

Amorçage de la campagne
	

Publier du contenu à une petite échelle ciblée en amont de la publication à grande échelle.
	

Tester l’efficacité du dispositif mis en place et affiner les messages (A/B testing, utilisation de black SEO, etc.)

​
	

Diffusion du contenu vers le grand public
	

Diffuser largement du contenu à l’ensemble du public   (diffusion de narratifs et artefacts sur les réseaux sociaux, publication   d’articles, rédaction de commentaires, etc.).
	

Atteindre les publics ciblés.

​
	

Maximisation de l’exposition 
	

Amplifier via des stratégies cross-plateformes, de flooding et via des réseaux de trolls ou de bots.
	

Assurer un maximum d’effets de la campagne.

​
	

Mise en œuvre d’actions agressives en ligne
	

Nuire à ses adversaires dans les espaces en ligne par le biais du harcèlement, de la divulgation d'informations privées et du contrôle de l'espace d'information.
	

Supprimer toute opposition et remise en question de la campagne.

​
	

Mise en œuvre d’actions hors ligne 
	

Inciter les utilisateurs à s’engager physiquement : de l’appel à manifester à l’achat de marchandises en passant par l’action violente.
	

Faire basculer la campagne virtuelle dans le monde réel et toucher de nouvelles audiences.

​
	

Persistance dans l’environnement informationnel
	

Poursuite del’amplification de narratifs et effacement des traces (dissimulation des moyens employés, de l’identité des acteurs), même si l’événement principal est terminé.
	

Assurer la continuité de la campagne sur le long-terme.

Évaluation
	

Évaluation de l’efficacité des actions
	

Évaluer l’efficacité des actions
	

​| Left-aligned | Center-aligned | Right-aligned |
| :---         |     :---:      |          ---: |
| git status   | git status     | git status    |
| git diff     | git diff       | git diff      |

 


Les techniques et procédures

Les techniques et procédures décrivent le comportement de l’acteur malveillant. Par sa projection sous forme de matrice dynamique, et sachant que chaque étape de la matrice s’appuie sur les étapes précédentes, la détection d’un TTPs permet de mettre en évidence l’état d’avancée d’une campagne, les moyens utilisés par l’attaquant et enfin, d’évaluer son niveau d’effort et les capacités dont il dispose.

Et ensuite ?

La fondation Disarm se charge de faire vivre cette matrice en mettant notamment à jour les TTPs en fonction de l’évolution et de la sophistication des campagnes. Elle travaille également à plébisciter son emploi auprès des différents acteurs de la lutte contre la désinformation : plateformes, institutions, think tanks, etc. afin, entre autres, de permettre une détection plus rapide, une qualification plus qualitative et un partage de l’information efficace.

La désinformation a besoin de se nourrir de la maturité et de l’avancée des questions qui préoccupent la CTI en matière d’état de la menace, d’anticipation, de capitalisation et de partage de l'information. Du lien entre ces deux matières peut naître des outils comme la matrice Disarm. C’est d’autant plus important qu’en retour, Disarm est un outil qui peut également alimenter le travail de la CTI. En effet, des campagnes comme Ghostwriter ou bien encore les Macron Leaks démontrent bien l’intrication de ces différents champs. Que ce soit le fruit d’une campagne mise en place par un acteur étatique ou une action opportuniste permise par une fuite de données, les pratiques des attaquants ne se limitent pas au champ du cyber ou de la désinformation. Ainsi doit-il en être également pour les acteurs qui souhaitent lutter contre. C’est la complémentarité technique de ces matrices qui permet la description de ces campagnes complexes. Et c’est la complémentarité des pratiques qui permettra une lutte efficace contre la désinformation et les campagnes informationnelles.



